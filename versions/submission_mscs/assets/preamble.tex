%
% This document is a preamble for 'reopn' project
%

% ==============================
% Load Packages
% ==============================

\usepackage{amsfonts}
\usepackage{amssymb}  
\usepackage{amsthm} 
\usepackage{amsmath} 
\usepackage{caption}
\usepackage[inline]{enumitem}
  \setlist{itemsep=0em, topsep=0em, parsep=0em}
  %\setlist[enumerate]{label=(\alph*)}
\usepackage{etoolbox}
\usepackage{stmaryrd} 
\usepackage[dvipsnames]{xcolor}
  \definecolor{myurlcolor}{rgb}{0.6,0,0}
  \definecolor{mycitecolor}{rgb}{0,0,0.8}
  \definecolor{myrefcolor}{rgb}{0,0,0.8}
\usepackage{graphicx}
  \graphicspath{ {assets/} }
\usepackage{mathtools}
\usepackage{tikz}
  \usetikzlibrary{matrix,
    arrows,
    shapes,
    decorations.markings,
    decorations.pathreplacing}
\usepackage{circuitikz}
  \usepackage{todonotes}

% ====================================
% Biblio
% ====================================
  
\usepackage[english]{babel}
\usepackage[utf8]{inputenc}
\usepackage{csquotes}% Recommended
\usepackage[style=authoryear-ibid,backend=biber]{biblatex}
\addbibresource{assets/biblio.bib}

\usepackage{hyperref}
\hypersetup{colorlinks,
	linkcolor=myrefcolor,
	citecolor=mycitecolor,
	urlcolor=myurlcolor}

% ====================================
% New Commands
% ====================================

%\renewcommand*{\backref}[1]{(Referred to on page #1.)}

% symbols ~~~~~~~~~~~~~~~~~~~~~~~~~~~
\renewcommand{\epsilon}{\varepsilon}
\newcommand{\op}{^{\scriptsize{ \textrm{op} } }}
\newcommand{\inv}{^{-1}}
\newcommand{\iso}{\cong}
\renewcommand{\equiv}{\simeq}
\newcommand{\bydef}{\coloneqq}
\newcommand{\hcirc}{\circ_{\textup{h}}}
\newcommand{\vcirc}{\circ_{\textup{v}}}
\renewcommand{\hat}{\widehat}


% categories ~~~~~~~~~~~~~~~~~~~~~~~~~
\newcommand{\A}{\cat{A}}
\newcommand{\B}{\cat{B}}
\newcommand{\C}{\cat{C}}
\newcommand{\D}{\cat{D}}
\newcommand{\E}{\cat{E}}
\renewcommand{\P}{\cat{P}}
\newcommand{\Q}{\cat{Q}}
\newcommand{\R}{\cat{R}}
\newcommand{\T}{\cat{T}}
\newcommand{\U}{\cat{U}}
\newcommand{\V}{\cat{V}}
\newcommand{\W}{\cat{W}}
\newcommand{\X}{\cat{X}}
\newcommand{\Y}{\cat{Y}}
\newcommand{\Z}{\cat{Z}}
\renewcommand{\AA}{\bicat{A}}
\newcommand{\BB}{\bicat{B}}
\newcommand{\CC}{\bicat{C}}
\newcommand{\DD}{\bicat{D}}
\newcommand{\EE}{\bicat{E}}
\newcommand{\PP}{\bicat{P}}
\newcommand{\QQ}{\bicat{Q}}
\newcommand{\RR}{\bicat{R}}
\newcommand{\TT}{\bicat{T}}
\newcommand{\UU}{\bicat{U}}
\newcommand{\VV}{\bicat{V}}
\newcommand{\WW}{\bicat{W}}
\newcommand{\XX}{\bicat{X}}
\newcommand{\YY}{\bicat{Y}}
\newcommand{\ZZ}{\bicat{Z}}
\newcommand{\AAA}{\dblcat{A}}
\newcommand{\BBB}{\dblcat{B}}
\newcommand{\CCC}{\dblcat{C}}
\newcommand{\DDD}{\dblcat{D}}
\newcommand{\EEE}{\dblcat{E}}
\newcommand{\MMM}{\dblecat{M}}
\newcommand{\PPP}{\dblcat{P}}
\newcommand{\QQQ}{\dblcat{Q}}
\newcommand{\RRR}{\dblcat{R}}
\newcommand{\SSS}{\dblcat{S}}
\newcommand{\TTT}{\dblcat{T}}
\newcommand{\UUU}{\dblcat{U}}
\newcommand{\VVV}{\dblcat{V}}
\newcommand{\WWW}{\dblcat{W}}
\newcommand{\XXX}{\dblcat{X}}
\newcommand{\YYY}{\dblcat{Y}}
\newcommand{\ZZZ}{\dblcat{Z}}

% common x-categories
\newcommand{\FinSet}{\cat{FinSet}}
\newcommand{\Set}{\cat{Set}}
\newcommand{\Rel}{\cat{Rel}}
\newcommand{\RRel}{\bicat{Rel}}
\newcommand{\FinHilb}{\cat{FinHilb}}
\newcommand{\Ab}{\cat{Ab}}
\newcommand{\Vect}{\cat{Vect}}
\newcommand{\Mod}{\cat{Mod}}
\newcommand{\Pos}{\cat{Pos}}
\newcommand{\FinGraph}{\cat{FinGraph}}
\newcommand{\Graph}{\cat{Graph}}
\newcommand{\RGraph}{\cat{RGraph}}
\newcommand{\Top}{\cat{Top}}
\newcommand{\TTop}{\bicat{Top}}
\newcommand{\Cat}{\cat{Cat}}
\newcommand{\CCat}{\bicat{Cat}}
\newcommand{\Bicat}{\cat{Bicat}}
\newcommand{\BBicat}{\bicat{Bicat}}
\newcommand{\DblCat}{\cat{DblCat}}
\newcommand{\DDblCat}{\bicat{DblCat}}
\newcommand{\Topos}{\cat{Topos}}
\newcommand{\TTopos}{\bicat{Topos}}
\newcommand{\Span}{\cat{Span}}
\newcommand{\SSpan}{\bicat{Span}}
\newcommand{\Csp}{\cat{Csp}}
\newcommand{\CCsp}{\bicat{Csp}}
\newcommand{\Gram}{\cat{Gram}}
\newcommand{\StrCsp}{\cat{StrCsp}}
\newcommand{\SStrCsp}{\bicat{StrCsp}}
\newcommand{\SSStrCsp}{\dblcat{S} \bicat{trCsp}}
\newcommand{\StrCspGram}{\cat{StrCspGram}}
\newcommand{\RRewrite}{\bicat{Rewrite}}
\newcommand{\RRRewrite}{\dblcat{R}\bicat{ewrite}}
\newcommand{\MMMonicSpCsp}[1]{\dblcat{M}\bicat{onicSp(Csp(#1}}
\newcommand{\ZX}{\cat{ZX}}
\newcommand{\ZZX}{\bicat{ZX}}
\newcommand{\ZZZX}{\dblcat{ZX}}
\newcommand{\FinGraphGamma}{\FinGraph \downarrow \Gamma}
\newcommand{\AdjTopos}{\cat{AdjTopos}}

% functors ~~~~~~~~~~~~~~~~~~~~~~~~~
\newcommand{\core}{\mathbf{core}}
\newcommand{\Lang}{\mathrm{Lang}}

% text formatting ~~~~~~~~~~~~~~~~~~~~~
\newcommand{\df}[1]{\textbf{#1}}
\newcommand{\cat}[1]{\mathsf{#1}}
\newcommand{\bicat}[1]{\mathbf{#1}}
\newcommand{\dblcat}[1]{\mathbb{#1}}
\newcommand{\type}[1]{\mathtt{#1}}
\newcommand{\Daniel}[1]{\textcolor{red}{(#1)}}

% arrows ~~~~~~~~~~~~~~~~~~~~~~~~~~~~~~
\newcommand{\from}{\colon}
\newcommand{\rel}{\nrightarrow}
\newcommand{\To}{\Rightarrow}
\newcommand{\xto}[1]{\xrightarrow{#1}}
\newcommand{\monicto}{\rightarrowtail}
\newcommand{\dderiv}[2]{#1 \rightsquigarrow #2}
\newcommand{\deriv}[2]{#1 \rightsquigarrow^\ast #2}
\renewcommand{\gets}{\leftarrow}
\newcommand{\monicgets}{\leftarrowtail}
\newcommand{\xgets}[1]{\xleftarrow{#1}}
\newcommand{\spn}[3]{#1 \gets #2 \to #3}
\newcommand{\xspn}[5]{#1 \xgets{#2} #3 \xto{#4} #5}
\newcommand{\csp}[3]{#1 \to #2 \gets #3}
\newcommand{\xcsp}[5]{#1 \xto{#2} #3 \xgets{#4} #5}
\newcommand{\lrto}{\rightleftarrows}

% math operators ~~~~~~~~~~~~~~~~~~~~~~~~~
\DeclareMathOperator{\Hom}{Hom}
\DeclareMathOperator{\id}{id}
\DeclareMathOperator{\im}{im}
\DeclareMathOperator{\Sub}{Sub}
\DeclareMathOperator{\colim}{colim}
\DeclareMathOperator{\ob}{ob}
\DeclareMathOperator{\arr}{arr}
\DeclareMathOperator{\edges}{edges}
\DeclareMathOperator{\nodes}{nodes}
\DeclareMathOperator{\Sh}{Sh}

% delimiters ~~~~~~~~~~~~~~~~~~~~~~~~~
%\renewcommand{\l(}{\left(}
%\newcommand{\r)}{\right)}

% =========================================
% Environments and Counters
% =========================================

\newtheorem{theorem}{Theorem}[section]
\newtheorem*{theorem*}{Theorem}
\newtheorem{lemma}[theorem]{Lemma}
\newtheorem{proposition}[theorem]{Proposition}
\newtheorem{corollary}[theorem]{Corollary}

\theoremstyle{remark}
\newtheorem{remark}[theorem]{Remark}
\newtheorem{notation}[theorem]{Notation}

\theoremstyle{definition}
\newtheorem{example}[theorem]{Example} 
\newtheorem{definition}[theorem]{Definition}

\setcounter{tocdepth}{1} % Sets depth for table of contents. 

% ==========================================
% Tikz Macros
% ==========================================

% arrows for commuting diagrams
\tikzset{
  cd/.style={
    ->,
    scale=6,
    >=angle 90,
    font=\scriptsize}}

% arrows for graphs
\tikzset{
  graph/.style={
    ->,
    scale=2,
    >=triangle 45,
    font=\scriptsize}}

% arrow head in middle of edge
\tikzset{->-/.style={decoration={%
      markings,
      mark=at position .5 with {\arrow{>}}},
      postaction={decorate}}}

% arrow head user-positioned
\tikzset{->-pos/.style={decoration={%
      markings,
      mark=at position #1 with {\arrow{>}}},
      postaction={decorate}}}

% arrow head in middle of edge
\tikzset{-|->/.style={decoration={%
      markings,
      mark=at position .5 with {\arrow{|}},
      mark=at position 1 with {\arrow{>}}},
    postaction={decorate}}}

% walking reflexive graph
\newcommand{\rgraph}[2]{%
  \begin{tikzpicture}[scale=0.75,baseline=-3pt]
    \node (a) at (0,0) {$ #1 $};
    \node (b) at (1,0) {$ #2 $};
    \draw [->]
    ([yshift= 4pt]a.east) to ([yshift= 4pt]b.west);
    \draw [->]
    ([yshift=-4pt]a.east) to ([yshift=-4pt]b.west);
    \draw [->]
    (b.west) to (a.east);
  \end{tikzpicture}
}

% walking graph
\newcommand{\graph}[2]{%
  \begin{tikzpicture}[scale=0.75,baseline=-3pt]
    \node (a) at (0,0) {$ #1 $};
    \node (b) at (1,0) {$ #2 $};
    \draw [->]
    ([yshift=4pt]a.east) to ([yshift=4pt]b.west);
    \draw [->]
    ([yshift=-4pt]a.east) to ([yshift=-4pt]b.west);
  \end{tikzpicture}
}

% open tipped arrow
\newcommand{\opento}[2]{%
  \begin{tikzpicture}[scale=0.75,baseline=-3pt]
    \node (a) at (0,0) {$ #1 $};
    \node (b) at (1,0) {$ #2 $};
    \draw [->, open triangle 60]
    (a.east) to (b.west);
  \end{tikzpicture}
}

% inline horizontal arrow
\newlength\mylen
\settowidth\mylen{$\to$}

\newcommand{\horarrow}{%
  \to\kern-0.55\mylen\vline height 1.2ex depth
  -0.4pt\kern0.55\mylen}

% adjunction
\newcommand{\adjunction}[4]{%
  \begin{tikzpicture}[baseline=-3pt]
    \node (1) at (0,0) {\( #1 \)};
    \node (2) at (2,0) {\( #4 \)};
    \draw [->]
    ([yshift= 4pt]2.west) to
    node [above] {\scriptsize{ $ #2 $ }}
    ([yshift= 4pt]1.east);
    \draw [->]
    ([yshift= -4pt]1.east) to
    node [below] {\scriptsize{ $ #3 $ }}
    node [above,yshift= -1.5pt] {\scriptsize{$ \perp $}}
    ([yshift= -4pt]2.west);
  \end{tikzpicture}
  % 
}
