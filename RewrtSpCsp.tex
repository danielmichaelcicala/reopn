\documentclass{amsart}          

% PACKAGES ~~~~~~~~~~~~~~~~~~~~

\usepackage{amsfonts}
\usepackage{amssymb}  
\usepackage{amsthm} 
\usepackage{amsmath} 
\usepackage{caption}
\usepackage[inline]{enumitem}
\setlist{itemsep=0em, topsep=0em, parsep=0em}
\setlist[enumerate]{label=(\alph*)}
\usepackage{etoolbox}
\usepackage{stmaryrd} 
\usepackage[dvipsnames]{xcolor}
\definecolor{editcolour}{rgb}{0.7,0.1,0}
\definecolor{hrefcolour}{rgb}{0,0,0.7}

\usepackage[]{hyperref}
\hypersetup{colorlinks,linkcolor={hrefcolour},citecolor={hrefcolour},urlcolor={hrefcolour}}
\usepackage{graphicx}
\graphicspath{ {img/} }
\usepackage{mathtools}

\usepackage{tikz}
\usetikzlibrary{matrix,arrows,shapes,decorations.markings,decorations.pathreplacing}
\usepackage{todonotes}

% NEW COMMANDS ~~~~~~~~~~~~~~~~~~

% mathbb
% \newcommand{\RR}{\mathbb{R}}
% \newcommand{\ZZ}{\mathbb{Z}}
% \newcommand{\NN}{\mathbb{N}}
% \newcommand{\QQ}{\mathbb{Q}}
% \newcommand{\CC}{\mathbb{C}}
% \newcommand{\FF}{\mathbb{D}}

% symbols
\renewcommand{\epsilon}{\varepsilon}
\newcommand{\op}{^{\scriptsize{ \textrm{op} } }}
\newcommand{\iso}{\cong}
\renewcommand{\equiv}{\simeq}

% categories
\newcommand{\A}{\cat{A}}
\newcommand{\B}{\cat{B}}
\newcommand{\C}{\cat{C}}
\newcommand{\D}{\cat{D}}
\newcommand{\E}{\cat{E}}
\renewcommand{\P}{\cat{P}}
\newcommand{\Q}{\cat{Q}}
\newcommand{\R}{\cat{R}}
\newcommand{\S}{\cat{S}}
\newcommand{\T}{\cat{T}}
\newcommand{\U}{\cat{U}}
\newcommand{\V}{\cat{V}}
\newcommand{\W}{\cat{W}}
\newcommand{\X}{\cat{X}}
\newcommand{\Y}{\cat{Y}}
\newcommand{\Z}{\cat{Z}}
\renewcommand{\AA}{\bicat{A}}
\newcommand{\BB}{\bicat{B}}
\newcommand{\CC}{\bicat{C}}
\newcommand{\DD}{\bicat{D}}
\newcommand{\EE}{\bicat{E}}
\newcommand{\PP}{\bicat{P}}
\newcommand{\QQ}{\bicat{Q}}
\newcommand{\RR}{\bicat{R}}
\newcommand{\SS}{\bicat{S}}
\newcommand{\TT}{\bicat{T}}
\newcommand{\UU}{\bicat{U}}
\newcommand{\VV}{\bicat{V}}
\newcommand{\WW}{\bicat{W}}
\newcommand{\XX}{\bicat{X}}
\newcommand{\YY}{\bicat{Y}}
\newcommand{\ZZ}{\bicat{Z}}
\newcommand{\AAA}{\dblcat{A}}
\newcommand{\BBB}{\dblcat{B}}
\newcommand{\CCC}{\dblcat{C}}
\newcommand{\DDD}{\dblcat{D}}
\newcommand{\EEE}{\dblcat{E}}
\newcommand{\PPP}{\dblcat{P}}
\newcommand{\QQQ}{\dblcat{Q}}
\newcommand{\RRR}{\dblcat{R}}
\newcommand{\SSS}{\dblcat{S}}
\newcommand{\TTT}{\dblcat{T}}
\newcommand{\UUU}{\dblcat{U}}
\newcommand{\VVV}{\dblcat{V}}
\newcommand{\WWW}{\dblcat{W}}
\newcommand{\XXX}{\dblcat{X}}
\newcommand{\YYY}{\dblcat{Y}}
\newcommand{\ZZZ}{\dblcat{Z}}

\newcommand{\Set}{\cat{Set}}
\newcommand{\Graph}{\cat{Graph}}
\newcommand{\RGraph}{\cat{RGraph}}
\newcommand{\Top}{\cat{Top}}
\newcommand{\Cat}{\cat{Cat}}
\newcommand{\Bicat}{\cat{Bicat}}
\newcommand{\DblCat}{\cat{DblCat}}
\newcommand{\Topos}{\cat{Topos}}
\newcommand{\Gram}{\cat{Gram}}
\newcommand{\Span}{\cat{Span}}
\newcommand{\Cospan}{\cat{Cospan}}

% text formatting
\newcommand{\defn}[1]{\textbf{#1}}
\newcommand{\cat}[1]{\mathrm{#1}}
\newcommand{\bicat}[1]{\mathbf{#1}}
\newcommand{\dblcat}[1]{\mathbb{#1}}
\newcommand{\type}[1]{\mathtt{#1}}
\newcommand{\edit}[1]{\textcolor{editcolour}{(#1)}}

% arrows
\newcommand{\from}{\colon}
\newcommand{\rel}{\nrightarrow}
\newcommand{\To}{\Rightarrow}
\newcommand{\xto}[1]{\xrightarrow{#1}}
\newcommand{\squigto}{\rightsquigarrow}
\newcommand{\xgets}[1]{\xleftarrow{#1}}
\newcommand{\tospan}{\xrightarrow{\mathit{sp}}}
\newcommand{\spn}[3]{#2 \to #1 \times #3}
\newcommand{\csp}[3]{#1 + #3 \to #2}
\newcommand{\tocospan}{\xrightarrow{\mathit{csp}}}

% OPERATORS ~~~~~~~~~~~~~~~~~~

\DeclareMathOperator{\Hom}{Hom}
\DeclareMathOperator{\id}{id}
\DeclareMathOperator{\im}{im}
\DeclareMathOperator{\Sub}{Sub}
\DeclareMathOperator{\colim}{colim}

% ENVIRONMENTS & COUNTERS ~~~~~~~~~~~

\newtheorem{theorem}{Theorem}[section]
\newtheorem{lemma}[theorem]{Lemma}
\newtheorem{proposition}[theorem]{Proposition}
\newtheorem{corollary}[theorem]{Corollary}

\theoremstyle{remark}
\newtheorem{remark}[theorem]{Remark}
\newtheorem{notation}[theorem]{Notation}

\theoremstyle{definition}
\newtheorem{example}[theorem]{Example} 
\newtheorem{definition}[theorem]{Definition}

\setcounter{tocdepth}{1} % Sets depth for table of contents. 

% TIKZ TYPES ~~~~~~~~~~~~~~~~~~~~~

% arrow head in middle of edge
\tikzset{->-/.style={decoration={%
      markings,
      mark=at position .5 with {\arrow{>}}},postaction={decorate}}
}

% arrow head user-positioned
\tikzset{->-pos/.style={decoration={%
      markings,
      mark=at position #1 with {\arrow{>}}},postaction={decorate}}
}

% arrow head in middle of edge
\tikzset{-|->/.style={decoration={%
      markings,
      mark=at position .5 with {\arrow{|}},mark=at position 1 with {\arrow{>}}},postaction={decorate}}
}

% INLINE DIAGRAMS ~~~~~~~~~~~~~~~

% walking reflexive graph
\newcommand{\rgraph}[2]{%
  \begin{tikzpicture}[scale=0.75,baseline=-3pt]
    \node (a) at (0,0) {$ #1 $};
    \node (b) at (1,0) {$ #2 $};
    \draw [->]
    ([yshift= 4pt]a.east) to ([yshift= 4pt]b.west);
    \draw [->]
    ([yshift=-4pt]a.east) to ([yshift=-4pt]b.west);
    \draw [->]
    (b.west) to (a.east);
  \end{tikzpicture}
}

% walking graph
\newcommand{\graph}[2]{%
  \begin{tikzpicture}[scale=0.75,baseline=-3pt]
    \node (a) at (0,0) {$ #1 $};
    \node (b) at (1,0) {$ #2 $};
    \draw [->]
    ([yshift=4pt]a.east) to ([yshift=4pt]b.west);
    \draw [->]
    ([yshift=-4pt]a.east) to ([yshift=-4pt]b.west);
  \end{tikzpicture}
}

% open tipped arrow
\newcommand{\opento}[2]{%
  \begin{tikzpicture}[scale=0.75,baseline=-3pt]
    \node (a) at (0,0) {$ #1 $};
    \node (b) at (1,0) {$ #2 $};
    \draw [->, open triangle 60]
    (a.east) to (b.west);
  \end{tikzpicture}
}

% inline horizontal arrow
\newlength\mylen
\settowidth\mylen{$\to$}

\newcommand{\horarrow}{%
  \to\kern-0.55\mylen\vline height 1.2ex depth
  -0.4pt\kern0.55\mylen}

% adjunction
\newcommand{\adjunction}[4]{%
  \begin{tikzpicture}[baseline=-3pt]
    \node (1) at (0,0) {( #1 )};
    \node (2) at (2,0) {( #4 )};
    \draw [->]
    ([yshift= 4pt]2.west) to
    node [above] {\scriptsize{ $ #2 $ }}
    ([yshift= 4pt]1.east);
    \draw [->]
    ([yshift= -4pt]1.east) to
    node [below] {\scriptsize{ $ #3 $ }}
    node [above,yshift= -1.5pt] {\scriptsize{$ \perp $}}
    ([yshift= -4pt]2.west);
  \end{tikzpicture}
  % 
}

\author{Daniel Cicala}
\title{Rewriting Structured Cospans}

% ~~~~~~~~~~~~~~~~~~~~~~~~~~~~~~~~~~~~~~~~
% 
% ~~~~~~~~~~~ begin document~~~~~~~~~~~~~~
% 
% ~~~~~~~~~~~~~~~~~~~~~~~~~~~~~~~~~~~~~~~~

\begin{document}
\maketitle{}

% ~~~~~~~~~~~~~~~~~~~~~~~~~~~~~~~~~~~~~~~~
% ~~~~~~~~~~~ introduction ~~~~~~~~~~~~~~~
% ~~~~~~~~~~~~~~~~~~~~~~~~~~~~~~~~~~~~~~~~

\section{Introduction}
\label{sec-1}
\subsection{Using structured cospans to introduce inductive rewriting on certain topoi}
\label{sec-1-1}
\subsection{generalizing gadducci.}
\label{sec-1-2}

% ~~~~~~~~~~~~~~~~~~~~~~~~~~~~~~~~~~~~~~~~
% ~~~~~~~~~~~ structured cospans~~~~~~~~~~
% ~~~~~~~~~~~~~~~~~~~~~~~~~~~~~~~~~~~~~~~~

\section{Structured Cospans}
\label{sec:StrCsp}

% ~~~~~~~~~~~~~~~~~~~~~~~~~~~
% ~~~~~~~~~~~~~~~~~~~~~~~~~~~

\subsection{Introduce the section}
\label{sec-2-1}

%____________________________________________________
\subsubsection{Background on structured cospans}
\label{sec-2-1-1}
\begin{enumerate}
\item decorated cospans
\label{sec-2-1-1-1}
\item structured cospans
\label{sec-2-1-1-2}
\end{enumerate}

Structured cospans were introduced to model compositional systems. In
fact, they were the second approach to model compositional systems
using cospans. Fong invented \emph{decorated cospans},
%
\todo{cite decorated cospans}
%
the first approach.  Structure cospans are
introduced here as well as by Baez and Courser.
%
\todo{cite baez-courser strcsp}
%
The latter has work two aims: maximize the
generality of the structured cospan construction using double
categories and also to compare decorated and structured cospans. Our
present interests are focused on introducing rewriting on structured
cospans, hence we will make a number of restrictions that Baez and
Courser do not.  These restriction are harmless, however, as most
cases of interest fall within the our parameters.

In this section, we will make explicit two perspectives on structured
cospans, both through the language of category theory.  The first is
looking at structured cospans as an object with morphisms between
them. The second is as a morphism between ``interfaces''.  It is the
latter perspective that encodes the compositional structure.  

\subsubsection{introduce some notation here}
\label{sec-2-1-2}
\begin{enumerate}
\item spans are x --> y \texttimes{} z and cospans too
\label{sec-2-1-2-1}
\end{enumerate}
\subsubsection{minimum assumptions at this point: working with a geom. morphism}
\label{sec-2-1-3}

For this section, fix an arbitrary geometric morphism
$ L \dashv R \from \X \to \A $. This is an adjunction
%
\[
  \adjunction{\X}{L}{R}{\A}
\]
%
between (elementary) topoi with $ L $ left exact. Because spans and
cospans factor heavily into this work, we use the notation
%
\(
 (f,g) \from \spn{x}{y}{z}
\)
% 
for a span
%
\[
  x \xgets{f} y \xto{g} z
\]
%
and
%
\(
  (f,g) \from \csp{x}{y}{z}
\)
% 
for a cospan
%
\[
  x \xto{f} y \xgets{g} z.
\]
% 
Because all of the
categories in this paper will have products and coproducts, this
notation is sensible.

% ~~~~~~~~~~~~~~~~~~~~~~~~
% ~~~~~~~~~~~~~~~~~~~~~~~~

\subsection{Structured cospans as objects}
\label{sec:StrCspAsObject}

\subsubsection{define the category StrCsp(L) of structured cospans as objects}
\label{sec-2-2-1}

\begin{definition}\label{df:strcsp}

  A \defn{ structured cospan } is a cospan of the form
  $ La \to x \gets Lb $.
  
\end{definition}

There is no novelty in a simple cospan, but we present this as a new
definition to place ourselves into the context of systems
modelling. Also, the reason for restricting ourselves to a geometric
morphism at this point is elusive. Especially so because the other
introducory work on structured cospans by Baez and Courser did not ask
for this. Let us say that this requirement arises from practical and
aesthetic considerations. For one, a nicer theory develops, but also
it is a sufficiently strong condition to introduce rewriting of
structured cospans.

\subsubsection{analogy to open systems and the role each part of the geom. morphism plays}
\label{sec-2-2-2}

Our motivations being the modeling of open systems, beginning
the story by fixing a geometric morphism might signal the abstract
nature of this work and obfuscate the connection between structured
cospans and open systems. So that we do not drift too far from our
concrete motivations, we will draw analogies between our construction
and open systems throughout.  

Here is the first such analogy. One should view the topos $ \A $ as
consisting of closed systems and their morphisms. By a \emph{closed
  system}, we mean a system that cannot interact with the outside
world. The topos $ \X $ should be thought to contain possible
interfaces for the closed systems. Equipping a closed system with an
interface provides an avenue for the system to now interact with
compatible elements of the outside world. The interface dictates
compatibility as we explore below.  Such a system is no longer closed,
and so we call it an \emph{open system}. The left adjoint $ L $ sends
these interfaces into $ \X $ so that they might interact with the
closed systems. The right adjoint $ R $ can be though of as returning
a closed system's largest possible interface. A word of caution: this
is an informal analogy and should only be used to gain an intuition
for the nature of structured cospans. Now, let us turn to the first of
two perspectives on structured cospans.

Through this perspective, a structured cospan consists of a closed
system $ x $ with an interface identified by the arrows from $ La $
and $ Lb $. Ignoring questions of causality, it is safe to consider
$ La $ as the input to $ x $ and $ Lb $ as the output. The sum total
of the closed system $ x $ equipped with interface $ La + Lb $ is our
open system. As expected, a morphism of open system ought to respect
these components.

\begin{definition} \label{df:morph-of-strcsp}

  A morphism from one structured cospan
  %
  \(
    \csp{La}{x}{Lb}
  \)
  %
  to another
  %
  \(
    \csp{Lc}{y}{Ld}
  \)
  % 
  is a triple of arrows $ ( f,g,h ) $ that fit into the commuting
  diagram
  \[
    \begin{tikzpicture}
      \node (1) at (-2,1) {\( La \)};
      \node (2) at (0,1) {\( x \)};
      \node (3) at (2,1) {\( Lb \)};
      \node (4) at (-2,0) {\( Lc \)};
      \node (5) at (0,0) {\( y \)};
      \node (6) at (2,0) {\( Ld \)};
      \draw [->] (1) to node [above] {\scriptsize{\(  \)}} (2);
      \draw [->] (3) to node [above] {\scriptsize{\(  \)}} (2);
      \draw [->] (4) to node [below] {\scriptsize{\(  \)}} (5);
      \draw [->] (6) to node [below] {\scriptsize{\(  \)}} (5);
      \draw [->] (1) to node [left] {\scriptsize{\( Lf \)}} (4);
      \draw [->] (2) to node [left] {\scriptsize{\( g \)}} (5);
      \draw [->] (3) to node [left] {\scriptsize{\( Lh \)}} (6);
    \end{tikzpicture}
  \]
\end{definition}

It is a simple exercise to show that structured cospans and their
morphisms form a category.  Denote this category by $ \StrCsp_L
$. 

\subsubsection{start running example: open/ranked graphs}
\label{sec-2-2-3}

\begin{enumerate}
\item mention gaducci and kissinger stuff here
\label{sec-2-2-3-1}
\end{enumerate}

\begin{example}[Open graphs] \label{ex:open-graphs}

  The field of network theory is intimately tied with graph theory
  \cite{networks}. A natural example of a structured cospan is an
  \emph{open graph}. While this notion is not new,
  %
  \todo{cite DuncanKiss and Gadd}
  %
  our infrastructure generalizes it.

  Let
  %
  \[
    \RGraph \coloneqq [ \rgraph{\bullet}{\bullet} , \Set ]
  \]
  % 
  be the category of (directed reflexive multi) graphs. There is an
  adjunction
  %
  \[
    \adjunction{\RGraph}{L}{R}{\Set}
  \]
  % 
  where $ Rx $ is the node set of graph $ x $ and $ La $ is the
  edgeless graph with node set $ a $. An \defn{open graph} is a cospan
  %
  \(
      \csp{La}{x}{Lb}
  \)
  % 
  for sets $ a $, $ b $, and graph $ x $. An illustrated example, with
  the reflexive loops suppressed, is
  %
  \[
    \begin{tikzpicture}
      %
      \begin{scope} % left graph
      \node (1) at (0,1) { \( \bullet \) };
      \node (2) at (0,0) { \( \bullet \) };
      \draw [rounded corners] (-0.5,-0.5) rectangle (0.5,1.5);
      \end{scope}
      %
      \begin{scope}[shift={(2,0)}] % center graph
      \node (1) at (0,1) {\( \bullet \)};
      \node (2) at (0,0) {\( \bullet \)};
      \node (3) at (1,0.5) {\( \bullet  \)};
      \node (4) at (2,0.5) {\( \bullet  \)};
      \draw [->-] (1) to (3);
      \draw [->-] (2) to (3);
      \draw [->-] (3) to (4);
      \draw [rounded corners] (-0.5,-0.5) rectangle (2.5,1.5);
      \end{scope}
      %
      \begin{scope}[shift={(6,0)}] % right graph
      \node (1) at (0,0.5) {\( \bullet \)};
      \draw [rounded corners] (-0.5,-0.5) rectangle (0.5,1.5);
      \end{scope}
      %
      \begin{scope} % graph morphisms
        \node (1) at (0.5,0.5) {};
        \node (2) at (1.5,0.5) {};
        \node (3) at (4.5,0.5) {};
        \node (4) at (5.5,0.5) {};
        \draw [->] (1) to (2);
        \draw [->] (4) to (3);
      \end{scope}
    \end{tikzpicture}
  \]
  % 
  The boxed items are graphs and the arrows between boxes are graph
  morphims defined as suggested by the illustration.  In total, the
  three graphs and two graph morphisms make up a single open graph.   
    
\end{example}

% what good is the open bit?

Having seen this example, it becomes more apparent about how open
systems can ``connect'' together. Given another open graph whose inputs
coincide with the outputs of the graph above, we can connect
the inputs and outputs together to create a new open graph. By passing
from graphs to open graphs, we are introducing
\emph{compositionality}. The category $ \StrCsp_{L} $ does not encode
the compositional structure, but we introduce a new category
$ \Csp_L $ in Section \ref{sec:strcsp-as-arrows} for this purpose
and delay further discussion along these lines until then. At present,
we concern ourselves with showing that $ \StrCsp_L $ is a topos in a
functorial way.  While interesting in itself, this fact allows us to
introduce rewriting systems to structured cospans.

\subsubsection{prove that StrCsp : [ * --> * , Topos ] --> Topos is functor}
\label{sec-2-2-4}
\begin{enumerate}
\item requires proving that StrCsp(L) is a topos using Artin gluing
\label{sec-2-2-4-1}
\end{enumerate}

\begin{theorem} \label{thm:strcsp-istopos}

  The category $ \StrCsp_L $ is a topos.
  
\end{theorem}

\begin{proof}

  The category $ \StrCsp_L $ constructed using the adjunction
  $ L \dashv R \from \X \to \A $ is equivalent to the category
  whose objects are cospans of form
  %
  \(
      \csp{a}{Rx}{b}
  \)
  % 
  and morphisms are triples $ ( f,g,h ) $ fitting into the commuting
  diagram
  %
  \[
    \begin{tikzpicture}
      \node (1) at (-2,1) {\( w \)};
      \node (2) at (0,1) {\( Ra \)};
      \node (3) at (2,1) {\( x \)};
      \node (4) at (-2,0) {\( y \)};
      \node (5) at (0,0) {\( Rb \)};
      \node (6) at (2,0) {\( z \)};
      \draw [->] (1) to  node [] {\scriptsize{\(  \)}} (2);
      \draw [->] (3) to node [] {\scriptsize{\(  \)}} (2);
      \draw [->] (4) to node [] {\scriptsize{\(  \)}} (5);
      \draw [->] (6) to node [] {\scriptsize{\(  \)}} (5);
      \draw [->] (1) to node [left] {\scriptsize{\( f \)}} (4);
      \draw [->] (2) to node [left] {\scriptsize{\( Rg \)}} (5);
      \draw [->] (3) to node [left] {\scriptsize{\( h \)}} (6); 
    \end{tikzpicture}
  \]
  % 
  This, in turn, is equivalent to the comma category
  $ ( \A \times \A \downarrow \Delta R ) $, where
  $ \Delta \from \A \to \A \times \A $ is the diagonal functor. But
  the diagonal functor is right adjoint to taking binary
  coproducts. That means $ \Delta R $ is also a right adjoint and,
  furthermore, that $ ( \A \times \A \downarrow \Delta R ) $ is an
  instance of Artin glueing,
  %
  \todo{CITE ARTIN GLUEING}
  hence a topos.
%
\end{proof}

Until now, the topos $ \StrCsp_L $ depended on an adjunction fixed at
the beginning of the section.  By letting the adjunction vary, we see
that $ \StrCsp_{(-)} $ is actually functorial. For the following theorem, denote by $ \Topos $ the category of
elementary topoi and geometric morphisms.

\begin{theorem} \label{thm:strcsp-isfunctorial}

  There is a functor
  %
  \[
    \StrCsp_{(-)} \from
    [ \bullet \to \bullet , \Topos ]
    \to
    \Topos
  \]
  % 
  defined by  
  \[
    \begin{tikzpicture}
      \begin{scope}
      \node (1) at (-1,1) {\( \X \)};
      \node (2) at (-1,-1) {\( \X' \)};
      \node (3) at (1,1) {\( \A \)};
      \node (4) at (1,-1) {\( \A' \)};
      \draw [->] (1.-60) to node [right] {\scriptsize{\( G \)}} (2.60);
      \draw [<-] (1.-120) to node [left] {\scriptsize{\( F \)}} (2.120);
      \draw [<-] (1.30) to node [above] {\scriptsize{\( L \)}} (3.150);  
      \draw [->] (1.-30) to node [below] {\scriptsize{\( R \)}} (3.-150);
      \draw [->] (2.30) to node [above] {\scriptsize{\( R' \)}} (4.150);
      \draw [<-] (2.-30) to node [below] {\scriptsize{\( L' \)}} (4.-150);      
      \draw [<-] (3.-60) to node [right] {\scriptsize{\( F' \)}} (4.60);
      \draw [->] (3.-120) to node [left] {\scriptsize{\( G' \)}}
      (4.120);
      \node (5) at (0,-1) {\scriptsize{\( \top \)}};
      \node (6) at (0,1) {\scriptsize{\( \perp \)}};
      \node (7) at (-1,0) {\scriptsize{\( \dashv \)}};
      \node (8) at (1,0) {\scriptsize{\( \vdash \)}};
      \end{scope}
     % 
      \begin{scope}[shift={(3,0)}]
      \node (1) at (0,0) { $\xmapsto{ \StrCsp_{(-)} }$ };
      \end{scope}
      %
      \begin{scope}[shift={(5.5,0)}]
      \node (1) [inner sep=0.1cm] at (0,0) {\( \StrCsp_{L} \)};
      \node (2) [inner sep=0.15cm] at (4,0) {\( \StrCsp_{L'} \)};
      \node (3) at (2,0) {\scriptsize{ \( \perp \) }};
      \draw [->]
        ([yshift= 4pt]1.east) to
        node [above] {\scriptsize{ \( \Theta \) }}
        ([yshift= 4pt]2.west);
      \draw [->]
        ([yshift= -4pt]2.west) to
        node [below] {\scriptsize{ \( \Theta' \) } }
        ([yshift= -4pt]1.east);  
      \end{scope}
    \end{tikzpicture}
  \]
  % 
  which is in turn given by
  %
  \[
    \begin{tikzpicture}
      \begin{scope}
      \node (1) at (-2,1) {\( La \)};
      \node (2) at (0,1) {\( x \)};
      \node (3) at (2,1) {\( Lb \)};
      \node (4) at (-2,0) {\( Lc \)};
      \node (5) at (0,0) {\( y \)};
      \node (6) at (2,0) {\( Ld \)};
      \draw [->] (1) to node [above] {\scriptsize{\( m \)}} (2);
      \draw [->] (3) to node [above] {\scriptsize{\( n \)}} (2);
      \draw [->] (4) to node [above] {\scriptsize{\( o \)}} (5);
      \draw [->] (6) to node [above] {\scriptsize{\( p \)}} (5);
      \draw [->] (1) to node [left] {\scriptsize{\( Lf \)}} (4);
      \draw [->] (2) to node [left] {\scriptsize{\( g \)}} (5);
      \draw [->] (3) to node [left] {\scriptsize{\( Lh \)}} (6);
      \end{scope}
      %
      \begin{scope}[shift={(3,0)}]
      \node (1) at (0,0.5) { $ \xmapsto{ \Theta } $ };
      \end{scope}
      %
      \begin{scope}[shift={(6.5,0)}]
      \node (1) at (-2,1) {\( L'G'a \)};
      \node (2) at (0,1) {\( Gx \)};
      \node (3) at (2,1) {\( L'G'b \)};
      \node (4) at (-2,0) {\( L'G'c \)};
      \node (5) at (0,0) {\( Gy \)};
      \node (6) at (2,0) {\( L'G'd \)};
      \draw [->] (1) to node [above] {\scriptsize{\( Gm \)}} (2);
      \draw [->] (3) to node [above] {\scriptsize{\( Gn \)}} (2);
      \draw [->] (4) to node [above] {\scriptsize{\( Go \)}} (5);
      \draw [->] (6) to node [above] {\scriptsize{\( Gp \)}} (5);
      \draw [->] (1) to node [left] {\scriptsize{\( L'G'f \)}} (4);
      \draw [->] (2) to node [left] {\scriptsize{\( Gg \)}} (5);
      \draw [->] (3) to node [left] {\scriptsize{\( L'G'h \)}} (6);  
      \end{scope}
    \end{tikzpicture}
  \]
  %
  and
  %
  \[
    \begin{tikzpicture}
      \begin{scope}
      \node (1) at (-2,1) {\( L'a' \)};
      \node (2) at (0,1) {\( x' \)};
      \node (3) at (2,1) {\( L'b' \)};
      \node (4) at (-2,0) {\( L'c' \)};
      \node (5) at (0,0) {\( y' \)};
      \node (6) at (2,0) {\( L'd' \)};
      \draw [->] (1) to node [above] {\scriptsize{\( m' \)}} (2);
      \draw [->] (3) to node [above] {\scriptsize{\( n' \)}} (2);
      \draw [->] (4) to node [above] {\scriptsize{\( o' \)}} (5);
      \draw [->] (6) to node [above] {\scriptsize{\( p' \)}} (5);
      \draw [->] (1) to node [left] {\scriptsize{\( L'f' \)}} (4);
      \draw [->] (2) to node [left] {\scriptsize{\( g' \)}} (5);
      \draw [->] (3) to node [left] {\scriptsize{\( L'h' \)}} (6);
      \end{scope}
      %
      \begin{scope}[shift={(3,0)}]
      \node (1) at (0,0.5) { $ \xmapsto{ \Theta' } $ };
      \end{scope}
      %
      \begin{scope}[shift={(6.5,0)}]
      \node (1) at (-2,1) {\( LF'a' \)};
      \node (2) at (0,1) {\( Fx' \)};
      \node (3) at (2,1) {\( LF'b' \)};
      \node (4) at (-2,0) {\( LF'c' \)};
      \node (5) at (0,0) {\( Fy' \)};
      \node (6) at (2,0) {\( LF'd' \)};
      \draw [->] (1) to node [above] {\scriptsize{\( Fm' \)}} (2);
      \draw [->] (3) to node [above] {\scriptsize{\( Fn' \)}} (2);
      \draw [->] (4) to node [above] {\scriptsize{\( Fo' \)}} (5);
      \draw [->] (6) to node [above] {\scriptsize{\( Fp' \)}} (5);
      \draw [->] (1) to node [left] {\scriptsize{\( LF'f' \)}} (4);
      \draw [->] (2) to node [left] {\scriptsize{\( Fg' \)}} (5);
      \draw [->] (3) to node [left] {\scriptsize{\( LF'h' \)}} (6);  
      \end{scope}
    \end{tikzpicture}
  \]  
\end{theorem}

\begin{proof}

  In light of Lemma \ref{thm:strcsp-istopos}, it suffices to show that
  $ \Theta \dashv \Theta' $ gives a geometric morphism.

  Denote the structured cospans
  %
  \[
    La \xto{m} x \xgets{n} Lb
  \]
  % 
  in $ \StrCsp_{ L } $ by $ \ell $ and  
  %
  \[
    L'a' \xto{ m'} x' \xgets{ n'} L'b'
  \]
  % 
  in $ \StrCsp_{ L' } $ by $ \ell' $,
  respectively. Also, denote the unit and counit for $F \dashv G$ by
  $ \eta $, $ \varepsilon $ and for $ F' \dashv G' $ by $ \eta' $, $
  \varepsilon' $.  The assignments
  %
  \begin{align}
    \left(
      ( f,g,h ) \from \ell \to \Theta' \ell'
      \right)
    & \mapsto
    \left(
      ( \varepsilon' \circ F'f , \varepsilon \circ Fg , \varepsilon'
      \circ F'h )
      \from \Theta \ell \to \ell'
      \right) \\
      %
      \left(
      ( f',g',h' ) \from \Theta \ell \to \ell'
      \right)
    & \mapsto
      \left(
      ( G'f' \circ \eta', Gg' \circ \eta , G'h' \circ \eta' )
      \from \ell \to \Theta' \ell'
      \right) \\
  \end{align}
  %
  give a bijection $ \hom ( \Theta \ell , \ell' ) \simeq \hom ( \ell ,
  \Theta' \ell' ) $. Moreover, it is natural in $ \ell $ and $ \ell'
  $. This rests on the natural maps $ \eta $, $ \varepsilon $, $ \eta'
  $, and $ \varepsilon' $. The left adjoint $ \Theta' $ preserves
  finite limits because they are taken pointwise and $ L $, $ F $, and
  $ F' $ all preserve finite limits.

\end{proof}

\subsubsection{the category of structured cospan categories is a functor category so the functors should be certain natural transformations. Does the current definition of a structured cospan functor do this?}
\label{sec-2-2-5}

Even though a structured cospan category is actually a topos, and its
information is drawn from a geometric morphism of topoi, the morphims
between structured cospan categories that we are interested in do not
involve geometric morphisms. Given a pair of geometric morphisms from
%
\[
  \adjunction{\X}{L}{R}{\A}
  \quad
  \t{ and }
  \quad
  \adjunction{\X}{L'}{R'}{\A'}
\]
% 
a \defn{structured cospan functor}
%
\[
  \StrCsp_{L} \to \StrCsp_{L'}
\]
% 
consists of a a pair of finitely continuous and cocontinuous functors
%
\todo{we actually only need F preserves monics, p.o.'s and p.b.'s and
  G p.b.'s}
%
$ F \from \X \to \X' $ and $ G \from \A \to \A' $ such that
$ FL=L'F $ and $ GR = R'F $.  While structured cospan categories and
their morphisms form a category, we leave it unnamed.

% ~~~~~~~~~~~~~~~~~~~~~~~~
% ~~~~~~~~~~~~~~~~~~~~~~~~

\subsection{Structured cospans as arrows}
\label{sec-2-3}

\subsubsection{this intros compositionality to structured cospans}
\label{sec-2-3-1}
\subsubsection{define the composition of structured cospans via pushout}
\label{sec-2-3-2}

We now turn to capturing the compositional structure that
truly motivates the invention of structured cospans.  To do this, we
shift perspectives of structured cospans as objects in $ \StrCsp_{L} $
to structured cospans as morphisms, which of course, are composable. 

\begin{definition} \label{def:strcsp-arr}
  
  Denote by $ \Csp_{L} $ the category that has the same objects as
  $ \A $ and structured cospans $ \csp{La}{x}{Lb} $ as arrows of
  type $ a \to b $.
  
\end{definition}

Note that composition is defined by pushout:
%
\[
  \begin{tikzpicture}
    \begin{scope}
    \node (1) at (0,0) {\( La \)};
    \node (2) at (1,0.5) {\( x \)};
    \node (3) at (2,0) {\( Lb \)};
    \node (4) at (3,0) {\( Lb \)};
    \node (5) at (4,0.5) {\( y \)};
    \node (6) at (5,0) {\( Lc \)};
    \node (7) at (2.5,0.25) {\( ; \)};
    \draw [->] (1) to node [] {\scriptsize{\(  \)}} (2);
    \draw [->] (3) to node [] {\scriptsize{\(  \)}} (2);
    \draw [->] (4) to node [] {\scriptsize{\(  \)}} (5);
    \draw [->] (6) to node [] {\scriptsize{\(  \)}} (5);
    \end{scope}
    %
    \begin{scope}[shift={(6,0)}]
    \node (1) at (0,0.25) {\( \xmapsto{\circ} \)};
    \end{scope}
    %
    \begin{scope}[shift={(7,0)}]
    \node (1) at (0,0) {\( La \)};
    \node (2) at (1.5,0.5) {\( x +_{Lb} y \)};
    \node (3) at (3,0) {\( Lc \)};
     \draw [->] (1) to node [] {\scriptsize{\(  \)}} (2);
    \draw [->] (3) to node [] {\scriptsize{\(  \)}} (2); 
    \end{scope}
  \end{tikzpicture}
\]
% 

Pushouts, in a sense, are a way of glueing things
together. Hence using pushouts as composition is a sensible way to model
system connection. Given two systems
%
\[
  \csp{La}{x}{Lb}
  \quad
  \t{ and }
  \quad
  \csp{Lb}{y}{Lc}
\]
% 
sharing a common interface $ Lb $, their composition is like
connecting at $ Lb $. To illustrate how the structured cospan
formalism allows us to connect systems together, we return to the open
graphs example.

\subsubsection{illustrate with open graphs}
\label{sec-2-3-3}

\begin{example} \label{ex:open-graph-as-arrow}

  The open graph
  % 
  \[
    \begin{tikzpicture}
       %
      \begin{scope} % left graph
      \node (1) at (0,1) { \( \bullet \) };
      \node (2) at (0,0) { \( \bullet \) };
      \draw [rounded corners] (-0.5,-0.5) rectangle (0.5,1.5);
      \end{scope}
      %
      \begin{scope}[shift={(2,0)}] % center graph
      \node (1) at (0,1) {\( \bullet \)};
      \node (2) at (0,0) {\( \bullet \)};
      \node (3) at (1,0.5) {\( \bullet  \)};
      \node (4) at (2,0.5) {\( {}_{j} \bullet  \)};
      \draw [->-] (1) to (3);
      \draw [->-] (2) to (3);
      \draw [->-] (3) to (4);
      \draw [rounded corners] (-0.5,-0.5) rectangle (2.5,1.5);
      \end{scope}
      %
      \begin{scope}[shift={(6,0)}] % right graph
      \node (1) at (0,0.5) {\( \bullet_{j} \)};
      \draw [rounded corners] (-0.5,-0.5) rectangle (0.5,1.5);
      \end{scope}
      %
      \begin{scope} % graph morphisms
      \node (1) at (0.5,0.5) {};
      \node (2) at (1.5,0.5) {};
      \node (3) at (4.5,0.5) {};
      \node (4) at (5.5,0.5) {};
      \draw [->] (1) to (2);
      \draw [->] (4) to (3);
      \end{scope}
      %
    \end{tikzpicture}
  \]
  % 
  can be composed with the open graph
   %
  \[
    \begin{tikzpicture}
       %
      \begin{scope} % left graph
      \node (1) at (0,0.5) { \( {}_{j} \bullet \) };
      \draw [rounded corners] (-0.5,-0.5) rectangle (0.5,1.5);
      \end{scope}
      %
      \begin{scope}[shift={(2,0)}] % center graph
      \node (1) at (0,0.5) {\( {}_{j} \bullet \)};
      \node (2) at (2,0) {\( \bullet \)};
      \node (3) at (2,0.5) {\( \bullet  \)};
      \node (4) at (2,1) {\( \bullet  \)};
      \draw [->-] (1) to (2);
      \draw [->-] (1) to (3);
      \draw [->-] (1) to (4);
      \draw [rounded corners] (-0.5,-0.5) rectangle (2.5,1.5);
      \end{scope}
      %
      \begin{scope}[shift={(6,0)}] % right graph
      \node (2) at (0,0) {\( \bullet \)};
      \node (3) at (0,0.5) {\( \bullet  \)};
      \node (4) at (0,1) {\( \bullet  \)};
      \draw [rounded corners] (-0.5,-0.5) rectangle (0.5,1.5);
      \end{scope}
      %
      \begin{scope} % graph morphisms
      \node (1) at (0.5,0.5) {};
      \node (2) at (1.5,0.5) {};
      \node (3) at (4.5,0.5) {};
      \node (4) at (5.5,0.5) {};
      \draw [->] (1) to (2);
      \draw [->] (4) to (3);
      \end{scope}
      %
    \end{tikzpicture}
  \]
  %
  to obtain
  %
  \[
    \begin{tikzpicture}
       %
      \begin{scope} % left graph
      \node (1) at (0,1) { \( \bullet \) };
      \node (2) at (0,0) { \( \bullet \) };
      \draw [rounded corners] (-0.5,-0.5) rectangle (0.5,1.5);
      \end{scope}
      %
      \begin{scope}[shift={(2,0)}] % center graph
      \node (1) at (0,1) {\( \bullet \)};
      \node (2) at (0,0) {\( \bullet \)};
      \node (3) at (1,0.5) {\( \bullet  \)};
      \node (4) at (2,0.5) {\( {}^{j} \bullet  \)};
      \node (5) at (3,0) {\( \bullet \)};
      \node (6) at (3,0.5) {\( \bullet  \)};
      \node (7) at (3,1) {\( \bullet  \)};
      \draw [->-] (1) to (3);
      \draw [->-] (2) to (3);
      \draw [->-] (3) to (4);
      \draw [->-] (4) to (5);
      \draw [->-] (4) to (6);
      \draw [->-] (4) to (7);
      \draw [rounded corners] (-0.5,-0.5) rectangle (3.5,1.5);
      \end{scope}
      %
      \begin{scope}[shift={(7,0)}] % right graph
      \node (2) at (0,0) {\( \bullet \)};
      \node (3) at (0,0.5) {\( \bullet  \)};
      \node (4) at (0,1) {\( \bullet  \)};
      \draw [rounded corners] (-0.5,-0.5) rectangle (0.5,1.5);
      \end{scope}
      %
      \begin{scope} % graph morphisms
      \node (1) at (0.5,0.5) {};
      \node (2) at (1.5,0.5) {};
      \node (3) at (5.5,0.5) {};
      \node (4) at (6.5,0.5) {};
      \draw [->] (1) to (2);
      \draw [->] (4) to (3);
      \end{scope}
      %
    \end{tikzpicture}
  \]
  %
  This composition glued the two open graphs together along the node $
  j $.
  
\end{example}

% ~~~~~~~~~~~~~~~~~~~~~~~~
% ~~~~~~~~~~~~~~~~~~~~~~~~

\subsection{A double category of structured cospans}
\label{sec:DblCatOfStrCsp}

Using double categories allows us to combine into a single instrument the
competing perspectives of structured cospans as objects and as
morphisms. In this section, we build two double categories: one to
combine our two perspectives on structured cospans and the other---
which comes in two flavors---anticipates the rewriting contained in
Section \ref{sec:rewriting-strcsp}. 

\subsubsection{hand wavey definition of double category. Point to shulman for definition}
\label{sec-2-4-1}

Because the ingredients for our constructions come from a geometric
morphism, there is an implicit cocartesian structure on the topoi
involved. This cocartesianness can be lifted up to the double
categories. For a precise definition of a symmetric monoidal double
category, we point to Shulman,
%
\todo{cite{shulman-constructing}}
%
though for the sake of completeness, we provide the key pieces. A
double category $ \CCC $ consists of a pair of categories
$ ( \C_0 , \C_1 ) $ with some additional data that binds them
together. The categories $ \C_0 $ and $ \C_1 $ are assemble together
into a double category as follows:
%
\begin{itemize}
\item the $ \CCC $-\emph{objects} are exactly $ \C_0 $-the objects,
\item the $ \CCC $-\emph{vertical arrows} $ c \to d $ between $ \C
  $-objects are exactly the $ \C_0 $ arrows, 
\item the $ \CCC $-\emph{horizontal arrows} $ c \horarrow d $
  between $ \CCC $-objects are the $ \C_1 $-objects together with some
  structure maps assigning the domain and codomain, and
\item the \emph{squares} of $ \CCC $ are
\[
  \begin{tikzpicture}
    \node (1) at (0,2) {\( c \)};
    \node (2) at (2,2) {\( d \)};
    \node (3) at (0,0) {\( c' \)};
    \node (4) at (2,0) {\( d' \)};
    \draw [-|->] (1) to node [above] {\scriptsize{\( m \)}} (2);
    \draw [->] (1) to node [left] {\scriptsize{\( f \)}} (3);
    \draw [->] (2) to node [right] {\scriptsize{\( g \)}} (4);
    \draw [-|->] (3) to node [below] {\scriptsize{\( n \)}} (4);
    \node (5) at (6,2) {\( c,c',d,d' \in \operatorname{ob} ( \C_0 ) \)};
    \node (6) at (6,1.33) {\( f,g \in \operatorname{arr} ( \C_0 ) \)};
    \node (7) at (6,.66) {\( m,n \in \operatorname{ob} ( \C_1 ) \)};
    \node (8) at (1,1) {\( \Downarrow \) \scriptsize{ \( \theta \)}};
    \node (9) at (6,0) {\( \theta \in \operatorname{arr} ( \C_1 ) \)};
  \end{tikzpicture}
\]
are the arrows of $ \C_1 $ together with structure maps attaching the
surrounding vertical arrows. 
\end{itemize}
%
The vertical arrows compose as they do in $ \C_0 $ and there is a
structure map for composing horizontal arrows. The $ 2 $-arrows can
compose both horizontally and vertically.

Observe that the horizontal arrows play a two role, as
objects in their origin category and arrows in the double
category. This reflects the content of the categories $ \StrCsp_{L} $
and $ \Csp_{L} $.

\subsubsection{Define double category SStrCsp:=[ob(A), A-arrows, StrCsps, StrCsp-arrows]}
\label{sec-2-4-3}

The first double category that we define is discussed in the
related work by Baez and Courser.
%
\todo{CITE COR 3.9 BAEZ COURSER STRCSP}
%
There, they prove that it actually is a double category, so we content
ourselves to simply provide the definition.

\begin{definition}

  There is a cocartesian double category
  $ \SSStrCsp \coloneqq ( \A , \StrCsp_{L} ) $ :
  \begin{itemize}
  \item the objects are all $ \A $-objects
  \item the vertical arrows $ a \to b $ all $ \A $-arrows, 
  \item the horizontal arrows $ \horarrow{a}{b} $ are
    $ \StrCsp\ob $-objects, that is cospans $ \csp{La}{x}{Lb} $, and
  \item the squares are commuting diagrams
    %
    \[
    \begin{tikzpicture}
    \node (1) at (0,1) {\( La \)};
    \node (2) at (1,1) {\( x \)};
    \node (3) at (2,1) {\( Lb \)};
    \node (4) at (0,0) {\( Lc \)};
    \node (5) at (1,0) {\( y \)};
    \node (6) at (2,0) {\( Ld \)};
    \draw [->] (1) to node [] {\scriptsize{\(   \)}} (2);
    \draw [->] (3) to node [] {\scriptsize{\(  \)}} (2);
    \draw [->] (4) to node [] {\scriptsize{\(  \)}} (5);
    \draw [->] (6) to node [] {\scriptsize{\(  \)}} (5);
    \draw [->] (1) to node [left] {\scriptsize{\( Lf \)}} (4);
    \draw [->] (2) to node [left] {\scriptsize{\( g \)}} (5);
    \draw [->] (3) to node [left] {\scriptsize{\( Lh \)}} (6);
    \end{tikzpicture}
  \]
  %
 \end{itemize}

  The tensor is pointwise application of the coproduct
  %
  \[
    \begin{tikzpicture}[scale=0.75]
      %
      \begin{scope}
      \node (1) at (0,1) {\( La \)};
      \node (2) at (1,1) {\( x \)};
      \node (3) at (2,1) {\( Lb \)};
      \node (4) at (0,0) {\( Lc \)};
      \node (5) at (1,0) {\( y \)};
      \node (6) at (2,0) {\( Ld \)};
      \draw [->] (1) to node [above] {\scriptsize{\(   \)}} (2);
      \draw [->] (3) to node [left] {\scriptsize{\(  \)}} (2);
      \draw [->] (4) to node [right] {\scriptsize{\(  \)}} (5);
      \draw [->] (6) to node [below] {\scriptsize{\(  \)}} (5);
      \draw [->] (1) to node [below] {\scriptsize{\(  \)}} (4);
      \draw [->] (2) to node [below] {\scriptsize{\(  \)}} (5);
      \draw [->] (3) to node [below] {\scriptsize{\(  \)}} (6);   
      \end{scope}
      %
      \begin{scope}[shift={(3,0)}]
      \node (1) at (0,0.5) {$ + $};  
      \end{scope}
      %
      \begin{scope}[shift={(4,0)}]
      \node (1) at (0,1) {\( Lw' \)};
      \node (2) at (1,1) {\( a' \)};
      \node (3) at (2,1) {\( Lx' \)};
      \node (4) at (0,0) {\( Ly' \)};
      \node (5) at (1,0) {\( b' \)};
      \node (6) at (2,0) {\( Lz' \)};
      \draw [->] (1) to node [above] {\scriptsize{\(   \)}} (2);
      \draw [->] (3) to node [left] {\scriptsize{\(  \)}} (2);
      \draw [->] (4) to node [right] {\scriptsize{\(  \)}} (5);
      \draw [->] (6) to node [below] {\scriptsize{\(  \)}} (5);
      \draw [->] (1) to node [below] {\scriptsize{\(  \)}} (4);
      \draw [->] (2) to node [below] {\scriptsize{\(  \)}} (5);
      \draw [->] (3) to node [below] {\scriptsize{\(  \)}} (6);   
      \end{scope}
      %
      \begin{scope}[shift={(7,0)}]
      \node (1) at (0,0.5) {$ \coloneqq $};  
      \end{scope}
      %
      \begin{scope}[shift={(8,0)}]]
      \node (1) at (0,1) {\( L ( w + w' ) \)};
      \node (2) at (2,1) {\( a + a' \)};
      \node (3) at (4,1) {\( L ( x + x' ) \)};
      \node (4) at (0,0) {\( L ( y + y' ) \)};
      \node (5) at (2,0) {\( b + b' \)};
      \node (6) at (4,0) {\( L ( z + z' ) \)};
      \draw [->] (1) to node [above] {\scriptsize{\(   \)}} (2);
      \draw [->] (3) to node [left] {\scriptsize{\(  \)}} (2);
      \draw [->] (4) to node [right] {\scriptsize{\(  \)}} (5);
      \draw [->] (6) to node [below] {\scriptsize{\(  \)}} (5);
      \draw [->] (1) to node [below] {\scriptsize{\(  \)}} (4);
      \draw [->] (2) to node [below] {\scriptsize{\(  \)}} (5);
      \draw [->] (3) to node [below] {\scriptsize{\(  \)}} (6);   
      \end{scope}
    \end{tikzpicture}
  \]  
\end{definition}

\subsubsection{This is the most natural structure to capture structured cospans. Bicategories are an alternative, but having the vertical arrows there is nice.}
\label{sec-2-4-2}

Double categories are a nice way of capturing both the object-ness and
arrow-ness of structured cospans.  An alternative would be to use
bicategories, but this doesn't reflect the nature of structured
cospans as faithfully as does double categories.

% ~~~~~~~~~~~~~~~~~~~~~~~~~~~~~~~~~~~~~~~~
% ~~~~~~~~~~~ rewriting ~~~~~~~~~~~~~~~~~~
% ~~~~~~~~~~~~~~~~~~~~~~~~~~~~~~~~~~~~~~~~

\section{Inductive Rewriting of Structured Cospans}
\label{sec-3}

\subsubsection{Introduce the section}
\label{sec-3-1}

\subsubsection{extra conditions on geom. morphism:}
\label{sec-3-1-1}
\begin{enumerate}
\item counit is monic
\label{sec-3-1-1-1}
\item in suboject algebra, (k=>LRk) U k = d
\label{sec-3-1-1-2}
\end{enumerate}

% ~~~~~~~~~~~~~~~~~~~~~~~~
% ~~~~~~~~~~~~~~~~~~~~~~~~

\subsection{Graph rewriting}
\label{sec-3-2}

\subsubsection{production of graphs}
\label{sec-3-2-1}
\subsubsection{graph grammars}
\label{sec-3-2-2}
\subsubsection{injective matching only}
\label{sec-3-2-3}
\subsubsection{pushout complement}
\label{sec-3-2-4}
\subsubsection{rewrite relation}
\label{sec-3-2-5}
\subsubsection{discrete rewrite relation = rewrite relation}
\label{sec-3-2-6}
\subsubsection{Gaducci's contrution and result}
\label{sec-3-2-7}

Graph rewriting is a well-studied field with many suitable references.
%
\todo{cite some shit}
%
To be self-contained, we present enough of the topic to facilitate a
connection with our main result.  

We start with the notion of a \emph{production}: a
span of graphs
%
\[
  \ell \monicgets k \monicto r
\]
%
with monic legs. This production, or ``rule'', states that whenever
we identify a copy of $ \ell $ in a graph $ g $, we can remove
that copy and replace it with $ r $.  By ``identifying a copy of $
\ell $ in $ g $'', we mean precisely that there is a mono $ m \from
\ell \to g $ called a \emph{matching morphism}.  The removal of $ \ell
$ from $ g $ is given by the \emph{pushout complelement} which is a
graph $ d $ that fits into a pushout diagram
%
\[
  \begin{tikzpicture}
    \node (1) at (0,1) {$ \ell $};
    \node (2) at (1,1) {$ k $};
    \node (3) at (0,0) {$ g $};
    \node (4) at (1,0) {$ d $};
    \draw [->] (1) to node [] {\scriptsize{$  $}} (2);
    \draw [->] (1) to node [left] {\scriptsize{$ m $}} (3);
    \draw [->] (3) to node [] {\scriptsize{$  $}} (4);
    \draw [->] (2) to node [] {\scriptsize{$  $}} (4);
  \end{tikzpicture}
\]
% 
It should be said that $ d $ need not exist, but when it does and
$ k \to \ell $ is monic, then $ d $ is unique up to isomorphism.

If $ P $ is a set of productions, often called a \emph{graph grammar},
there is an induced relation $ \squigto $ on the objects of the category
$ \RGraph $. This is defined by $ g \squigto h $ whenever there is a
production $ k \to \ell \times r $ in $ P $, a match $ \ell \to g $,
and a pushout complement $ d $ that fit into the double pushout
diagram
%
\[
  \begin{tikzpicture}
    \node (1) at (0,1) {$ \ell $};
    \node (2) at (1,1) {$ k $};
    \node (3) at (2,1) {$ r $};
    \node (4) at (0,0) {$ g $};
    \node (5) at (1,0) {$ d $};
    \node (6) at (2,0) {$ h $};
    \draw [->] (2) to node [] {\scriptsize{$  $}} (1);
    \draw [->] (2) to node [] {\scriptsize{$  $}} (3);
    \draw [->] (5) to node [] {\scriptsize{$  $}} (4);
    \draw [->] (5) to node [] {\scriptsize{$  $}} (6);
    \draw [->] (1) to node [] {\scriptsize{$  $}} (4);
    \draw [->] (2) to node [] {\scriptsize{$  $}} (5);
    \draw [->] (3) to node [] {\scriptsize{$  $}} (6);
    \node () at (0.5,0.5) {\scriptsize{p.o.}};
    \node () at (1.5,0.5) {\scriptsize{p.o.}};
  \end{tikzpicture}
\]
%
This content of the diagram is that we have identified a subgraph of
shape $ \ell $ inside of $ g $, removed this subgraph and replaced it
in a coherent way with $ r $, resulting in the graph $ h $. If
$ g \squigto h $, we say that $ h $ is a \emph{direct derivation} of
$ g $. The term ``direct'' refers to the derivation happening in a
single step.  In general, $ \squigto $ is not transitive, meaning that
this relation does not tell us about multi-step derivations. It is
also not reflexive, which is preferable for a nicer theory.
Therefore, it is customary to work with the so-called \emph{rewrite
  relation}, that is the reflexive and transitive closure
$ \squigto^\ast $. Indeed, each graph grammar $ P $ induces a unique
rewrite relation.

In term rewriting, there are two different viewpoints. The
\emph{operational} viewpoint can be concisely described as performing
substition. That is, we rewrite a term by substituting subterms with
other terms according to a rule. This is the viewpoint of the graph
rewriting discussed above.  On the other hand, the \emph{inductive}
viewpoint consists of building a class of rewritings from a collection
of basic ones.  This is the viewpoint that Gadducci and Heckel first
brought to graph rewriting in
%
\todo{cite gadd/heckle inductive}
%
and we introduce to adhesive rewriting in .
%
\todo{reference main theorme}
%

In proving their result, Gadducci and Heckel made use of the following
well-known fact: the graph grammars
%
$ { k_\alpha \to \ell_\alpha \times r_\alpha } $
%
and
%
$ { ( k_\alpha )_{\textup{disc.}  \to \ell_\alpha \times r_\alpha } $,
% 
where $ ( k_\alpha )_{\textup{disc.} $ is the discrete graph underlying $ k_\alpha $, induce the same rewrite relation.
%
\todo{cite erhig here}
%
Lemma
%
\todo{refence generalization lemma}
%
generalizes this result into our context.  They leveraged this fact to
construct a computad associate to a graph grammar $ P $. This computad
has underlying category $ \Csp_L $ where $ L $ is taken from Example \ref{ex:open-graphs}. We additionally add a 2-cell
%
\[
  \begin{tikzpicture}
    \node (1) at (0,0) {$ L \emptyset $};
    \node (2) at (2,0) {$ Lk $};
    \node (3) at (1,0) {\scriptsize{$ \Downarrow \gamma $}};
    \draw [->, bend left] (1.45) to node [] {\scriptsize{$ \hat{\ell} $}} (2.135);
     \draw [->, bend right] (1.-45) to node [] {\scriptsize{$ \hat{r} $}} (2.-135);
  \end{tikzpicture}
\]
% 
into the computad for every production $ k \to \ell \time r $ in $ P
$. By $ \hat{\ell} $, we mean the open graph
%
\(
  L \emptyset + Lk \to \ell
\)
% 
and the analogous open graph for $ \hat{r} $.  Note the use of the
discrete graph underlying $ k $.  Here, we translated Gadducci and
Heckel's work into the language of structured cospans.  This computad
then freely generates a 2-category which we call $ \cat{G} $, the
1-arrows of which are open graphs. Now, the hom-category
$ \cat{G}( L \emptyset , L \emptyset ) $ has objects the usual graphs
having no inputs or outputs. This hom-category captures the rewriting
relation as we state here.
%
\todo{cite gadd/heck induct thm 23 on the thm}
%

\begin{theorem}
  Let $ P $ be a graph grammar. Then $ g \squigto^\ast h $ if and only
  if there is a 2-arrow from
  %
  \(
    L \emptyset + L \emptyset \to g 
  \)
  % 
  to
  %
  \(
   L \emptyset + L \emptyset \to h
  \)
  % 
  in $ \cat{G} ( L \emptyset , L \emptyset ) $.  
\end{theorem}

Having shared a viewpoint on inductive graph rewriting, we now turn
to apply these ideas to inductive adhesive rewriting using structured cospans.








% ~~~~~~~~~~~~~~~~~~~~~~~~
% ~~~~~~~~~~~~~~~~~~~~~~~~

\subsection{Adhesive rewriting}
\label{sec-3-3}

\subsubsection{adhesive categories}
\label{sec-3-3-1}
\begin{enumerate}
\item Van Kampen condition
\label{sec-3-3-1-1}
\end{enumerate}
\subsubsection{productions}
\label{sec-3-3-2}
\subsubsection{adhesive grammars}
\label{sec-3-3-3}
\begin{enumerate}
\item category Gram
\label{sec-3-3-3-1}
\end{enumerate}
\subsubsection{injective matching}
\label{sec-3-3-4}
\subsubsection{pushout complement}
\label{sec-3-3-5}
\subsubsection{Derivations is functorial Gram --> Gram}
\label{sec-3-3-6}
\begin{enumerate}
\item idempotent
\label{sec-3-3-6-1}
\item identity on Gram is subfunctor of derivation
\label{sec-3-3-6-2}
\end{enumerate}
\subsubsection{Rewrite relation in terms of derivation functor}
\label{sec-3-3-7}
\subsubsection{The language functor and its relation to the rewrite relation}
\label{sec-3-3-8}

% ~~~~~~~~~~~~~~~~~~~~~~~~
% ~~~~~~~~~~~~~~~~~~~~~~~~

\subsection{Rewriting structured cospans}
\label{sec-3-4}

\subsubsection{StrCsp(L) < topoi < adhesive to can do rewriting}
\label{sec-3-4-1}
\subsubsection{Define the subcategory of Gram of Structured Cospans Grammars StrGram.}
\label{sec-3-4-2}
\subsubsection{Define Structured Cospan Language functor StrGram --> DblCat}
\label{sec-3-4-3}
\begin{enumerate}
\item Uses the double category MonRewrite in which to generate sub-double cats.
\label{sec-3-4-3-1}
\end{enumerate}
\subsubsection{thm: production and discrete production have same rewrite relation}
\label{sec-3-4-4}
\subsubsection{thm: rewrite relation same as squares from 0 to 0.}
\label{sec-3-4-5}


% ~~~~~~~~~~~~~~~~~~~~~~~~
% ~~~~~~~~ examples ~~~~~~
% ~~~~~~~~~~~~~~~~~~~~~~~~

\subsection{Examples}
\label{sec-3-5}
\subsubsection{Encoding petri nets into dpo rewrites (see bib folder for Graph Rewriting w Petri Nets)}
\label{sec-3-5-1}



% ~~~~~~~~~~~~~~~~~~~~~~~~~~~~~~~~~~~~~~~~
% 
% ~~~~~~~~~~~ bibliography ~~~~~~~~~~~~~~~
% 
% ~~~~~~~~~~~~~~~~~~~~~~~~~~~~~~~~~~~~~~~~

\begin{thebibliography}{99}
  % use APA style
  % \bibitem{1st-citation}
  
\bibitem{Cic_SpCsp}
D.$\backslash$ Cicala. Spans of cospans. \emph{Theory Appl.\ Categ.\} 33, Paper No.$\backslash$ 6, 131--147. 2018.

\bibitem{CicCour_SpCspTopos}
D.$\backslash$ Cicala and K.$\backslash$ Courser. Spans of cospans in a topos. \emph{Theory Appl.\ Categ.} 33, Paper No.$\backslash$ 1, 1--22. 2018.

\bibitem{DixKiss_OpenGraphs}
L.$\backslash$ Dixon, and A.$\backslash$ Kissinger. Open-graphs and monoidal theories. \emph{Math.\ Structures Comput.\ Sci.\}, \textbf{23}, no.$\backslash$ 2, 308--359. 2013.

\bibitem{Ehrig_GraphGram}
H.$\backslash$ Ehrig, M.$\backslash$ Pfender, and H.J.$\backslash$ Schneider. Graph-grammars: An algebraic approach. In \emph{Switching and Automata Theory, 1973. SWAT'08. IEEE Conference Record of 14th Annual Symposium on}, 167--180. IEEE. 1973.

\bibitem{Fong_DecorCsp}
B.$\backslash$ Fong. Decorated cospans. \emph{Theory Appl.\ Categ.\} 30, Paper No.$\backslash$ 33, 1096--1120. 2015.
          
\bibitem{Gadd_IndGraphTrans}
F.$\backslash$ Gadducci, R.$\backslash$  Heckel. An inductive view of graph transformation. \emph{Recent trends in algebraic development techniques}, 223--237, Lecture Notes in Comput. Sci., 1376, Springer, Berlin. 1998.

\bibitem{LackSobo_Adhesive}
S.$\backslash$ Lack, and P.$\backslash$ Soboci$\backslash$'\{n\}ski. Adhesive categories. In \emph{International Conference on Foundations of Software Science and Computation Structures}, 273--288. Springer, Berlin, Heidelberg. 2004.

\bibitem{Wraith_ArtinGlue}
G.$\backslash$ Wraith. Artin glueing. \emph{J.\ Pure Appl.\ Algebra} \textbf{4}, 345--348. 1974.

\end{thebibliography}

\end{document}
