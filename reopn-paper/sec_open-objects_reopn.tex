%
% rewriting open objects
% open objects
%

% --- comment out when compiling ---
\input{file_preamble_reopn} % input preamble
%\bibliography{} % input bibliograph	

\section{Open objects} % section title
\label{sec:OpenObjects}

Our definition of an open graph is ripe for generalization.  In the present section, we start by defining \emph{open objects} and constructing a pair of categories emphasizing them.  We discover that the open graphs arises as a special case of the open objects.  The main result of this section is that the category of open objects and their morphisms form a topos.  

\begin{df} \label{df:OpenObject}
	Fix a functor $ \partial \from \A \to \X $ and suppose that $ \X $ has chosen pushouts. A \textbf{$ \partial $-open object}, or \textbf{open object} when $ \partial $ is understood, is a cospan of the form $ \partial a \to x \gets \partial a' $.  We will refer to $ \A $ as the category of \emph{boundary types} and to $ \X $ as the category of \emph{object types}.
\end{df}

\begin{remark}
	It is not yet apparent why we ask to pushout object types. This will play an important role in connecting open objects together, as we did with open graphs.
\end{remark}

\begin{remark}
	It may not seem that this construction warrants a definition, as we are merely renaming a cospan.  A fairer portrayal is not that we are renaming a cospan, but that we are reframing a cospan. Calling such a thing an ``open object'' places us into a network theory context, whereas ``cospan'' lacks such connotative powers.  There is a precedence for this in category theory, particularly in naming functors of a certain sort.  For example, a topological quantum field theory is defined to be a functor of a certain sort.  
\end{remark}

The idea behind an open object $ \partial a \to x \gets \partial a' $ is that $ x $ is the ``closed'' object which we open to interaction with other open objects via its \emph{inputs} $ \partial a $ and \emph{outputs} $ \partial a' $.  Specifically, the way in which a pair of open objects can interact is by fusing together to form a new open object.  This is modeled by setting open objects as arrows in a category.  The fusion of two open objects into one is then composition.  But composing cospans requires pushouts, so we now only 

\begin{df} 
	Let $ \X $ be a category with chosen pushouts and $ \partial \from \A \to \X $ a functor. Denote by $ \Csp{\partial} $ the category whose objects are those of $ \A $ and whose arrows $ a \to b $ are cospans $ \partial a \to x \gets \partial b $.
\end{df}

Chosen pushouts allow us to compose in $ \Csp{\partial} $:
\todo{input diagram D3-dCsp composition}
Compare this to composition in the category, mentioned above, whose morphisms are open graphs.

\begin{ex}
	Let $ \partial \from \Set \to \RGraph $ be the discrete graph functor. A $ \partial $-open object is exactly an open graph.  Also, we mentioned $ \Csp{\partial} $ in Section \ref{sec:MotivatingExample} as the category whose arrows are open graphs.  
\end{ex}










