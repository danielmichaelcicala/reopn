%
%	rewriting open objects
%	style
%	

%
%  packages
%

\usepackage{amsfonts}
\usepackage{amssymb}  
\usepackage{amsthm} 
\usepackage{amsmath} 
\usepackage{caption}
\usepackage[inline]{enumitem}
	\setlist{itemsep=0em, topsep=0em, parsep=0em}
	\setlist[enumerate]{label=(\alph*)}
\usepackage{doi}
\usepackage{etoolbox}
\usepackage[]{hyperref}
%	\definecolor{hyperrefcolor}{rgb}{0,0,0.7}
	\hypersetup{colorlinks,linkcolor={blue},citecolor={blue},urlcolor={blue}}
\usepackage{graphicx}
	\graphicspath{ {images/} }
\usepackage{mathtools}
\usepackage[numbers]{natbib}
\usepackage{stmaryrd} 
\usepackage{subcaption}
\usepackage{subfiles}
\usepackage{tikz}
	\usetikzlibrary{matrix,arrows,shapes,decorations.markings,decorations.pathreplacing}
\usepackage{todonotes}
\usepackage{url}
\usepackage{xcolor}

%
% commands
%

\newcommand{\RR}{\mathbb{R}}
\newcommand{\ZZ}{\mathbb{Z}}
\newcommand{\NN}{\mathbb{N}}
\newcommand{\QQ}{\mathbb{Q}}
\newcommand{\CC}{\mathbb{C}}
\newcommand{\DD}{\mathbb{D}}
\newcommand{\MM}{\mathbb{M}}
\renewcommand{\epsilon}{\varepsilon}

\newcommand{\Set}{\cat{Set}}
\newcommand{\Graph}{\cat{Graph}}
\newcommand{\RGraph}{\cat{RGraph}}
\newcommand{\Top}{\cat{Top}}
\newcommand{\Cat}{\cat{Cat}}
\newcommand{\A}{\cat{A}}
\newcommand{\B}{\cat{B}}
\newcommand{\C}{\cat{C}}
\newcommand{\X}{\cat{X}}
\newcommand{\Y}{\cat{Y}}
\newcommand{\Z}{\cat{Z}}
\newcommand{\core}[1]{\mathbf{core}(#1)}

\newcommand{\defn}[1]{\textbf{#1}}
\newcommand{\op}[1]{\operatorname{#1}}
\newcommand{\cat}[1]{\mathbf{#1}}
\newcommand{\dblcat}[1]{\mathbb{#1}}
\renewcommand{\t}[1]{\text{#1}}


\newcommand{\from}{\colon}
\newcommand{\xto}[1]{\xrightarrow{#1}}
\newcommand{\sm}{\smallsetminus}
\newcommand{\tospan}{\xrightarrow{\mathit{sp}}}
\newcommand{\tocospan}{\xrightarrow{\mathit{csp}}}
\newcommand{\diagram}[1]{\raisebox{-0.5\height}{\includegraphics{#1}}}

\newcommand{\Sp}[1]{\mathbf{Sp}(#1)}
\newcommand{\MonSp}[1]{\mathbf{MonSp}(#1)}
\newcommand{\SSp}[1]{\mathbb{S}\mathbf{p}(#1)}
\newcommand{\Csp}[1]{\mathbf{Csp}(#1)}
\newcommand{\CCsp}[1]{\mathbb{C}\mathbf{sp}(#1)}
\newcommand{\SpSp}[1]{\mathbf{Sp}(\mathbf{Sp}(#1))}
\newcommand{\SSpSp}[1]{\mathbb{S}\mathbf{p(\mathbf{Sp}(#1))}}
\newcommand{\CspCsp}[1]{\mathbf{Csp}(\mathbf{Csp}(#1))}
\newcommand{\CCspCsp}[1]{\mathbb{C}\mathbf{sp}(\mathbf{Csp}(#1))}
\newcommand{\MonSpCsp}[1]{\mathbf{MonicSp}(\mathbf{Csp}(#1))}
\newcommand{\MMonSpCsp}[1]{\mathbb{M}\mathbf{onicSp}(\mathbf{Csp}(#1))}
\newcommand{\EpCspSp}[1]{\mathbf{EpicCsp}(\mathbf{Csp}(#1))}
\newcommand{\EEpCspSp}[1]{\mathbb{E}\mathbf{picCsp}(\mathbf{Sp}(#1))}
\newcommand{\SpCsp}[1]{\mathbf{Sp}(\mathbf{Csp}(#1))}
\newcommand{\SSpCsp}[1]{\mathbb{S}\mathbf{p}(\mathbf{Csp}(#1))}

\newcommand{\FuncCsp}[1]{ #1 \t{-} \mathbf{Csp}}
\newcommand{\OpenOb}[1]{ #1 \t{-} \mathbf{Open} }
\newcommand{\Rewrite}[1]{ #1 \t{-} \mathbf{Rewrite} }
\newcommand{\RRewrite}[1]{ #1 \t{-} \mathbb{R}\mathbf{ewrite} }
\newcommand{\MonRewrite}[1]{ #1 \t{-} \mathbf{MonRewrite} }
\newcommand{\MMonRewrite}[1]{ #1 \t{-} \mathbb{M}\mathbf{on}\mathbb{R}\mathbf{ewrite} }

%
% math operators
%

\DeclareMathOperator{\Hom}{Hom}
\DeclareMathOperator{\id}{id}
\DeclareMathOperator{\ob}{Ob}
\DeclareMathOperator{\arr}{arr}
\DeclareMathOperator{\im}{im}
\DeclareMathOperator{\Aut}{Aut}
\DeclareMathOperator{\Bij}{Bij}
\DeclareMathOperator{\Sub}{Sub}
\DeclareMathOperator{\colim}{colim}

%
% envirnments and counters
%

\newtheorem{thm}{Theorem}[section]
\newtheorem{lem}[thm]{Lemma}
\newtheorem{prop}[thm]{Proposition}
\newtheorem{cor}[thm]{Corollary}

\theoremstyle{remark}
	\newtheorem{remark}[thm]{Remark}
	\newtheorem{notation}[thm]{Notation}

\theoremstyle{definition}
	\newtheorem{ex}[thm]{Example} 
	\newtheorem{df}[thm]{Definition}

% \setcounter{tocdepth}{1} % Sets depth for table of contents. 

%
% tikz types
%
\tikzset{->-/.style={decoration={%
			markings,
			mark=at position .5 with {\arrow{>}}},postaction={decorate}}
}
\tikzset{->-pos/.style={decoration={%
			markings,
			mark=at position #1 with {\arrow{>}}},postaction={decorate}}
}
\tikzset{->-/.style={decoration={%
			markings,
			mark=at position .5 with {\arrow{>}}},postaction={decorate}}
}
\tikzset{->-pos/.style={decoration={%
			markings,
			mark=at position #1 with {\arrow{>}}},postaction={decorate}}
}

%
% inline diagrams
%
\newcommand{\rgraph}[2]{%
	$\begin{tikzpicture}
	\node (a) at (0,0) {$ #1 $};
	\node (b) at (1,0) {$ #2 $};
	\draw [->] (a.30) to (b.150);
	\draw [->] (a.-30) to (b.-150);
	\draw [->] (b) to (a);
	\end{tikzpicture}$
}
\newcommand{\graph}[2]{%
	$\begin{tikzpicture}
	\node (a) at (0,0) {$ #1 $};
	\node (b) at (1,0) {$ #2 $};
	\draw [->] (a.30) to (b.150);
	\draw [->] (a.-30) to (b.-150);
	\end{tikzpicture}$
}